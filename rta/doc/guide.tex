% rtadoc.tex
\documentclass[twoside,a4paper,dvips]{report}
\usepackage{t1enc}
\usepackage{alltt}
\usepackage{changebar}
\bibliographystyle{plain}

\newcommand{\sem}[1]{\textit{#1}}
\newcommand{\res}[1]{\textbf{#1}}

\newcommand{\optionlabel}[1]{\mbox{\texttt{#1}}\hfil}
\newenvironment{option}%
  {\begin{list}{}%
         {\renewcommand{\makelabel}{\optionlabel}%
          \setlength{\labelwidth}{35pt}%
          \setlength{\leftmargin}{\labelwidth}%
          \addtolength{\leftmargin}{\labelsep}%
         }%
  }%
  {\end{list}}

\newcommand{\msg}[1]%
  {\vspace{\baselineskip}%
   \noindent\texttt{#1}%
   \newline\noindent}

% ------------------------------------------------------------------
\begin{document}
% --------- Front matter -------------------------------------------

\title{{\Huge\textbf{RTA User's Guide}}\\
\rule{\textwidth}{3pt}\\
{\normalsize{RTA: Response Time Analyzer --- 
Revision 1.2}}}

\date{\today}

\author{Juan A. de la Puente\\
        Departamento de Ingenieria de Sistemas Telem�ticos\\
        Universidad Polit�cnica de Madrid}

\maketitle

% copyright page

\pagestyle{empty}

\noindent 
\copyright Copyright 1998, Juan Antonio de la Puente 

\vspace{\baselineskip}

\noindent
RTA is free software; you can redistribute it and/or modify it under
terms of the GNU General Public License as published by the Free
Software Foundation; either version 2, or (at your option) any later
version. RTA is distributed in the hope that it will be useful, but
WITHOUT ANY WARRANTY; without even the implied warranty of MERCHANT
ABILITY or FITNESS FOR A PARTICULAR PURPOSE. See the GNU General
Public License for more details. You should have received a copy of
the GNU General Public License along with RTA; see file COPYING. If
not, write to the Free Software Foundation, 59 Temple Place - Suite
330, Boston, MA 02111-1307, USA.

\vspace{\baselineskip}

\noindent
Copies of this manual may be freely made and distributed, provided that
the copyright notice, this permission, and the appendix entitled ``GNU
General Public License'' are included exactly as in the original, and
if any modifications are made the derived work is distributed exactly
under the terms of a permission notice identical to this one.

\vspace{\baselineskip}

\begin{raggedright}
\textbf{Revision history}\\[6pt]
\begin{tabular}{lcp{0.75\textwidth}}
1.1 & 1998-05-20 & Initial version\\
1.2 & 1998-09-30 & 
     Task set file format enhanced to include information on resources
       used by tasks.

     Automatic computation of task priorities, priority ceilings for 
       shared objects, and worst case blocking times for tasks
    \\
\end{tabular}
\end{raggedright}


% table of contents 

\cleardoublepage
\pagestyle{plain}
\pagenumbering{roman}
\tableofcontents

% --------- Body of the document -----------------------------------


\cleardoublepage
\pagestyle{headings}
\pagenumbering{arabic}
\setcounter{page}{1}

% -------------------------------------------------------------------
\chapter{Introduction}

%\section{About RTA}

RTA is a software tool that performs response time analysis for
real-time systems.  The analysis is based on fixed-priority scheduling
theory \cite{Audsley&95}, and includes most of the techniques which
are commonly known as \emph{Rate-Monotonic Analysis}
\cite{Klein&93}. Before using RTA you should be acquainted with at
least the basics of fixed priority analysis theory as described
in e.g. \cite{Klein&93} or \cite{Burns&96}.

RTA is portable. Is has been written in Ada~95 using only the standard
libraries, and has no dependence on any underlying environment. It can
thus be compiled and installed with any valid Ada~95 compiler. In order
to keep the software simple and portable, only a plain text, command-line
style user interface is provided. Future versions of RTA may include
graphic edition or additional analysis tools that can be used with the
basic tool.

RTA is free software, and can be copied or modified under the terms of
the GNU General Public License. You should have received a copy of the
GNU General Public License distributed with RTA; see file COPYING.  If
not, write to the Free Software Foundation, 59 Temple Place - Suite
330, Boston, MA 02111-1307, USA.
                     
RTA has been developed by Juan Antonio de la Puente at the Department
of Telematic Systems Engineering of the Technical University of Madrid
(DIT/UPM).  Its development has been partially funded by CICYT, the
Spanish Board for Scientific and Technical Research under project
TIC96-0614.


%---------------------------------------------------------------------
\chapter{Running RTA}

\section{Getting started\label{sec:start}}

You can run RTA from any text shell on any machine where it has been
properly installed (see appendix \ref{ch:installation} if you need to
install it first.) Change to the \verb'examples' directory directly
under the \verb'rta' directory, and type

\begin{verbatim}
   rta example.tsf
\end{verbatim}

The parameter \verb'example.tsf' is the name of an example
\emph{task set file} (see \ref{sec:tasks} and \ref{sec:files} below.)
A task set file includes a description of the real-time attributes of
the tasks that make up the system to be analysed (their
\emph{profiles}). 
\begin{changebar}
RTA assigns priorities to the tasks in deadline-monotonic order, and
priority ceilings to all shared resources according to the priority
ceiling protocol. It then computes the response times for all the tasks,
and prints the results on your terminal as
shown in figure \ref{fg:example-out}.
\end{changebar}

\begin{figure}[htbp]
\begin{changebar}
\begin{center}

\begin{minipage}{\textwidth}
\footnotesize
\begin{verbatim}
Response time analysis for task set Sample
------------------------------------------
Id Task      A PR  Period  Offset  Jitter    WCET   Block Deadline Response Sch
-- --------- - -- ------- ------- ------- ------- ------- -------- -------- ---
 1 Task_3    P  3  30.000   0.000   0.000   8.000   2.000   30.000   10.000 Yes
 2 Task_2    P  2  40.000   0.000   0.000   6.000   0.000   40.000   14.000 Yes
 3 Task_1    P  1  50.000   0.000   0.000  19.000   0.000   50.000   47.000 Yes

Priority ceilings for shared resources
--------------------------------------
Id Name      PR
-- --------- --
 1 Lock_1     3
 2 Lock_2     2

Total processor utilization :  79.67%
\end{verbatim}
\end{minipage}

\caption{Output of the example run.\label{fg:example-out}}

\end{center}
\end{changebar}

\end{figure}

\begin{changebar}
The \verb'examples' directory includes further task set files, which
you can also analyse with the RTA program. %
See the comments on the files for an explanation of their particular
characteristics.
\end{changebar}

\section{Tasks and task sets\label{sec:tasks}}

Response time analysis is performed on \emph{task sets}. A task set is
a finite set of concurrently executing tasks with timing attributes.
There is no limit, other than imposed by the amount of available
storage, on the number of tasks a task set can have.

The attributes of each task are put together in its \emph{task profile}.
A task profile is a data structure with the following elements:

\begin{itemize}
\item The task \emph{name};

\item Its \emph{activation pattern}. The currently supported
activation patterns are: 

\begin{itemize}
\item \emph{Periodic}: the task is released at regular intervals of time.
\begin{changebar}
\item \emph{Sporadic}: the task is released at irregular instants, in
response to an \emph{event} which is signalled by some other
task. There is a minimum separation between two consecutive releases
of the same task.
\item \emph{Interrupt}: the task is released at irregular instants, in
response to external \emph{events}, which are signalled by
interrupts. There is a minimum separation between two consecutive
releases of the same task.
\end{changebar}
\end{itemize}

\item The task \emph{priority}. Tasks are assumed to be scheduled
according to their priorities, with the highest priority task
executing first. Priorities are positive numbers, with the higher
number denoting the higher priority.

The current version of RTA requires that each task has a different
priority.

\begin{changebar}
Priorities can be computed as part of the RTA execution (this is the
default), or manually entered in the input task set file, when the
\verb'-p' option is given (see \ref{sec:options}.)
\end{changebar}

\item The activation \emph{period\/} for periodic tasks, or the
minimum \emph{separation} between consecutive releases for sporadic
tasks.

\item The activation \emph{offset}. This value is usually set to zero,
except for periodic tasks that are released with an offset with
respect to the initial time (see \ref{sec:model}.)

\item The worst case activation \emph{jitter}. This value is mostly
relevant to sporadic tasks that are activated by other tasks or by the
reception of a message on a communication channel.

\item The worst case \emph{computation time\/} of the task over a
single activation cycle.

The computation time of a task must not be greater than its period.

\item The worst case \emph{blocking time\/} caused by priority
inversion from lower priority tasks (see \ref{sec:model}.)

\begin{changebar}
Blocking times can be computed as part of the RTA execution (this is
the default), or can be entered in the task set file when the
\verb'-b' option is given (see \ref{sec:options}.)
\end{changebar}

\item The worst case \emph{interference\/} caused by higher priority
tasks (see \ref{sec:computing}.) This value is computed by the RTA
tool as part of the response time computation.

\item The task \emph{deadline}, that is the time by which the task
activity is due to end, relative to the release time.

The deadline of a task can have any value, even if it is greater than
the task period.

\item The worst case \emph{response time\/}, as computed by the RTA
tool (see \ref{sec:computing}.)

\end{itemize}

\begin{changebar}
The task set file also contains the attributes of the shared resources
(protected by mutual exclusion \emph{locks}) which are used by two or
more tasks each. The attributes of a shared resource are put together
in its \emph{lock profile}. A lock profile is a data structure with
the following elements:
\begin{itemize}
\item The lock \emph{name\/} (usually the same as the shared resource
it protects.)
\item Its \emph{priority ceiling}, which equals the priority of the
highest priority task that can lock the resouurce (see
\ref{sec:model}).  Priority ceilings are positive numbers, with the
higher number denoting the higher priority.

Priority ceilings can be computed as part of the RTA execution (this is the
default), or manually entered in the input task set file, when the
\verb'-c' option is given (see \ref{sec:options}.)
\end{itemize}

In order to properly compute resoruce priority ceilings and task
blocking times, the task set file also includes a list of the shared
resources used by each task.

The task set data, including the individual task % and lock profiles,
are read by the RTA program from a \emph{task set file}. For each
task, at least the name, activation pattern, period, computation time,
and deadline, must be specified. The rest of the values can be set to
zero.
\end{changebar}

\section{Format of task set files\label{sec:files}}

Task set files are text files, containing characters in the ISO-8859
character set, which can be edited with any plain text editor. They
can also be automatically generated by static analysis tools from
source code or design models.\footnote{The current version of RTA does
not include any such tools.}

It is recommended that task set files be given names ending with
\verb'".tsf"', although this is not required by the tool.

Task set files may contain any number of blank characters or blank
lines.

\begin{changebar}
Comments start with \verb'"--"' and run through the end of the line.

A task set file begins with a declaration of the task set name, and the
number of tasks and locks that it contains.  This is followed by a
list of task profiles and a list of lock profiles.  The file ends with
an \verb'end' statement.

\begin{alltt}
  task_set_file ::=
     \res{task set} \sem{set}_name \res{with} \sem{task}_number \res{tasks}
     [ \res{and} \sem{lock}_number \res{locks} ]
  \res{is}
         \{lock_profile\}
         \{task_profile\}
  \res{end} \sem{set}_name \res{;}
\end{alltt}
\end{changebar}

Notice the semicolon terminating the \verb'end' statement.

\begin{changebar}
There must be exactly \sem{task}\_number task profiles and, if
provided, \sem{lock}\_number lock profiles.%
\footnote{If no lock number is provided, zero is assumed.}

A lock profile includes the name of the resource, and an optional
priority ceiling valuen in parentheses:

\begin{alltt}
   lock_profile ::=
      \res{lock} \sem{lock}_name [\res{(}\sem{priority}_number\res{)}];
\end{alltt}

Notice the semicolon at the end of the lock profile.

A task profile includes the name of the task, its activation pattern,
and a list of parameters separated by commas in parentheses. This may
be followed by a list of locks used by the task:

\begin{alltt}
   task_profile ::=
      \res{task} \sem{task_}name \res{is} activation_pattern \res{(}
         \sem{priority_}number,
         \sem{period_}time, \sem{offset_}time, \sem{jitter_}time,
         \sem{computation_}time, \sem{blocking_}time,
         \sem{deadline_}time, \sem{response_}time\res{)}
         [\res{uses} \sem{lock}_name \{, \sem{lock}_name\}];

   activation_pattern ::=
      periodic | sporadic | interrupt |undefined 
\end{alltt}
\end{changebar}

All the parameters must be present, even those valued zero.
Notice also the semicolon at the end of the task profile.

Names are strings beginning with a letter followed by any number of
alphanumeric, '\_'. '-', or '.' characters.  Case is not significant,
either for reserved words or names. Thus, \verb'task_1',
\verb'Task_1', and \verb'TASK_1' are equivalent.

Number values are written as unsigned integer numbers with no intervening
spaces, commas, underscores or any other characters.

Time values are entered as unsigned real numbers, optionnally
including a dot as a decimal point character, but no spaces, commas,
underscores or any other characters. There is no predefined time unit,
i.e. time values may represent seconds, milliseconds, or any other
time unit, provided it is used consistently for the whole task set. In
most cases you will probably find practical to use milliseconds as the
time unit.

Figure \ref{fg:example-tsf} shows the contents of the
\verb'example.tsf' task set file (the input to the sample run in
\ref{sec:start}.) You may find it useful to look at the other task set
files included in the \verb'examples' directory.

\begin{figure}[htbp]
\begin{changebar}
\begin{center}
\begin{minipage}{0.8\textwidth}\small
\begin{verbatim}
-- Sample task set
task set Sample with 3 tasks and 2 locks is
   -- locks
   lock Lock_1;
   lock Lock_2;
   -- tasks
   task Task_1 is periodic (0, 50, 0, 0, 19, 0, 0, 50, 0);
   task Task_2 is periodic (0, 40, 0, 0,  6, 0, 0, 40, 0)
      uses Lock_1 (2), Lock_2 (5);
   task Task_3 is periodic (0, 30, 0, 0,  8, 0, 0, 30, 0)
      uses Lock_1 (5);
end Sample;
\end{verbatim}
\end{minipage}

\caption{Example task set file (\texttt{example.tsf}).\label{fg:example-tsf}}

\end{center}
\end{changebar}
\end{figure}

\section{Output}

The results of the response time analysis are printed on the standard
output (usually the user's terminal, unless redirected to a file). The
output format is shown on figure \ref{fg:example-out}.  The output
produced by the RTA program includes the task set name, the task
profile attributes for each task,\footnote{In the current version of
RTA the interference value is not printed in order to save space.}
and an indication of the task schedulability.\footnote{A task is
schedulable if its response time is less than or equal to its
deadline.}
\begin{changebar}
The tasks are sorted by priority, unless otherwise
specified by the \verb'-n' option (see \ref{sec:options}). The
resource locks are also listed, together with thier respective
priority ceiling values.  Finally, the total processor utilization of
the task set is printed.
\end{changebar}

The results are not output if an error has been found in the task set
file, or the total processor utilisation is above 100\%. In the latter
case the response time computation algorithm does not converge (see
\ref{sec:computing}), and the task set is not schedulable.

You can redirect the output of RTA to another file using the \verb'-o'
option (see \ref{sec:options}) or the shell redirection mechanism, if
there is one available.

\section{Command line format\label{sec:options}}

The full RTA comand line format is:

\begin{alltt}
   rta [-\textit{flags}] \textit{input_file} [-s \textit{save_file}] [-o \textit{output_file}]
\end{alltt}

\noindent The meaning of the command line arguments is as follows.

\begin{option}
\item [\textit{flags}] is a string including one or more of the
following flag characters:
\begin{changebar}
   \begin{option}
   \item [v] Verbose. If this flag is set, the program writes 
   messages on the standard error file indicating progress in
   execution. This flag is normally used only for debugging purposes.

   \item [h] Help. The program prints a help message on the the
   standard error file.

   \item [p] Do not assign priorities to tasks, but instead use 
   the priority values provided in the task set file.

   \item [c] Do not compute priority ceilings for resource locks,
   but instead use the values provided in the task set file.

   \item [c] Do not compute task blocking times, but instead use the
   values provided in the task set file.

   \item [u] Update. The input task set file is replaced by a new file
   includig the time attributes (interference and response time) which
   have been computed as part of the analysis.

   \item [n] Do not sort the task set file. By the default, both tasks
   and resource locks are sorted by priorities.

   \end{option}
\end{changebar}

Notice that all the flags must be entered in a single string
preceded by a hyphen. Only the \verb'-s' and \verb'-o' options are
entered separately.

\item [\textit{input\_file}] is the name of a task set file which is taken as
input for the analysis.

This argument is mandatory, except when the \verb'v' or \verb'h' are
specified and no further arguments or flags are given. In this case,
only the version number or, respectively, help message (see
\ref{sec:errors}) is printed.

\item [-s \textit{save\_file}] Save the task set resulting from the
analysis, including the computed priority, interference, and response
time values, into a file named \textit{save\_file}.

No attempt is made to check whether the file already exists. If this
is the case, the file is overwritten and the previous contents is
lost.

\item [-o \textit{output\_file}] Redirect the output file to the
specified file.

No attempt is made to check whether the file already exists. If this
is the case, the file is overwritten and its previous contents is
lost.

\end{option}

\section{Messages\label{sec:errors}}

The RTA program may issue different kinds of messages to the user. All
the messages are shown on the standard error file, which is usally set
to the user's terminal, unless redirected.

There three kinds of messages: information, warning, and error
messages. All of them have been designed in order to be easily
understandable.

\subsection{Information messages}

Information messages are shown whenever the \verb'v' flag is set, in
order to show progress of execution through the different parts of the
RTA program.

A help message is hown when the \verb'-h' flag is set, or when there
is an error in the command line syntax.

\subsection{Warning messages}

Warning messages are shown to inform the user of problems that are not
fatal, so that analysis is still possible, although perhaps with
degraded performance or incorrect results.

\subsection{Error messages}

Error messages are shown to inform the user that a problem has been
found that makes analysis impossible to be performed.

Some error messages may require special actions. These are:

\msg{   Error: could not read input file \textit{name}}
\msg{   Error: could not update file \textit{name}}
\msg{   Error: could not save on file \textit{name}}
\msg{   Error: could not write results\textit{name}}

All of the above are input-output error messages that are output when
there is some problem with the named files. Possible error causes are:
\begin{itemize}
\item Input errors: The file does not exist or is corrupt.
\item Output errors: The file cannot be created or there is no available
 space on disk.
\end{itemize}

\msg{   Error: unknown error}
An unexpected error has occurred during RTA execution. When this
happens, submit a bug report with all the required information.

\chapter{How RTA works}


\section{Computation model\label{sec:model}}

The current version of response time analysis assumes a computation
model consisting of a set of concurrent tasks running on a
monoprocessor computer. The scheduling method is preemptive priority
based \cite{Burns&94b}.%
\footnote{See \cite{Burns&96} for a detailed description
of the principles and algorithms involved in response time analysis.}%
All tasks must have a distinct integer priority, with higher values
representing higher priorities.

Tasks may periodic or sporadic. Periodic tasks are released at times
given by
\[ t_{r}(k) = k \cdot T + O \]
or
\[ t_{r}(k) = t_{r}(k-1) + T ; t_{r}(0) = O \]
where $T$ is the task \emph{period} and $O$ is the initial
\emph{offset}.  

The actual time at which a task is activated may be delayed by the
duration of a variable \emph{jitter}, bounded by a maximum value
$J$. The usal source of jitter for periodic tasks is \emph{tick
scheduling}. This is a common scheduling method which is based on a
clock issuing ticks at regular intervals. The scheduler is activated
on every clock tick in order to check which periodic tasks have become
active. If the period of a task is not a multiple of the tick period,
the task may suffer from jitter, with $J = T$. All the tasks that have
periods which are multiples of the tick period have zero jitter.

Sporadic tasks are asociated to internal or external events, and are
released at the times the event occurs. The release times of a
sporadic task verify
\[ t_{r}(k) \geq t_{r}(k-1) + T \]
In order to keep the system temporal behaviour predictable, a minimum
\emph{separation\/} between consecutive releases, $T$, is specified.

Sporadic tasks may also have jitter, which ususally comes from
communication delays when a task is activated by other task by means
of a message.

Task communication is assumed to be based on shared memory protected
by \emph{locks}. Critical sections must be executed according to some
protocol that ensures bounded blocking time, such as the
\emph{priority inheritance protocol\/} (PHP), the \emph{priority
ceiling protocol\/} (PCP), or the \emph{immediate priority ceiling
protocol\/} (IPCP), also known as the \emph{highest locker protocol\/}
(HLP) \cite{Sha&90a}.%
\footnote{Examples of this kind of communication are Ada protected
objects \cite{Ada95} and POSIX.1c mutexes \cite{POSIX95}.}
Conditional communication is limited to those cases in which the
duration of waiting on a condition is bounded.

\begin{changebar}
Worst case execution times (WCET) are assumed to be known for all the
tasks.  The maximum blocking times due to priority inversion on shared
objects or conditional waiting can be computed in the following way
when the PCP, IPCP or HLP protocol is used:
\begin{equation}
B_{i} = \max_{j \in \mathit{LP}(i), k \in \mathit{HC}(i)} C_{j,k}
\end{equation}
where $\mathit{LP}(i)$ is the set of all tasks with priority lower
than $i$, $\mathit{HC}(i)$ is the set of all locks with priority
ceiling higher or equal than $i$, and $C_{j,k}$ is the computation
time of the longest critical section executed by task $j$ on lock $k$.
\end{changebar}

The WCET and blocking time of a task are denoted by $C$ and $B$,
respectively.%

All tasks are assumed to have \emph{deadlines}, denoted by $D$. A task
is \emph{schedulable\/} if its worst case response time is not greater
than its deadline. Deadlines can be arbitrarily long.

%\section{Run-time system parameters}

\section{Computing response times\label{sec:computing}}

The worst case response time of task $\tau_{i}$, $R_{i}$, depends on
three components:
\begin{itemize}
\item The worst case computation time of  $\tau_{i}$, $C_{i}$
\item The maximum blocking suffered by $\tau_{i}$, $B_{i}$
\item The \emph{interference\/} $I_{i}$, caused by preemption of
$\tau_{i}$ by higher priority tasks.
\end{itemize}

The first two components are assumed to be known. The interference
on $\tau_{i}$ over an interval $(0,t]$ is

\begin{equation}
I_{i}(t) = \sum_{j \in \mbox{hp}(i)} \left\lceil \frac{t + J_{j}}{T_{j}}
\right\rceil C_{j}
\end{equation}
where $\mbox{hp}(i)$ is the set of tasks with priorities higher than
$\tau_{i}$.

The response time is
\begin{equation}\label{eq:rperiodic}
R_{i} = (q+1) C_{i} + B_{i} + I_{i}(R_{i}) - q T_{i} + J_{i}
\end{equation}
for periodic tasks, and
\begin{equation}\label{eq:rsporadic}
R_{i} = (q+1) C_{i} + B_{i} + I_{i}(R_{i}) - q T_{i}
\end{equation}
for sporadic tasks, where $q+1$ is the number of releases of
$\tau_{i}$ in the interval $(0,R_{i}]$. 

Since both $q$ and $R_{i}$ are unknown, and appear on the rigt side of
equations \ref{eq:rperiodic} and \ref{eq:rsporadic}, an iterative
algorithm is used to compute the response time. For $q = 0, 1, 2,
\ldots$, the following recurrence relationship is applied:

\begin{equation}
w_{i}^{n+1}(q) = (q+1)C_{i} + B_{i} + \sum_{j \in \mbox{hp}(i)} 
\left\lceil \frac{w_{i}^{n} + J_{j}}{T_{j}} \right\rceil C_{j}
\end{equation}

A convenient starting value for $w_{i}^{n}(q)$ is

\begin{equation}
 w_{i}^{0}(q) = (q+1)C_{i} + B_{i} 
\end{equation}

The iteration stops when a value $w_{i}^{\ast}(q)$ is found such that

\[ w_{i}^{n+1}(q) = w_{i}^{n}(q) = w_{i}^{\ast}(q) \]

Then we have
\begin{equation}
 R_{i}(q) = w_{i}^{\ast}(q) -qT_{i} + J_{i} 
\end{equation}
for periodic tasks, and
\begin{equation}
 R_{i}(q) = w_{i}^{\ast}(q) -qT_{i}
\end{equation}
for sporadic tasks. 

The value of $q$ is incremented until a value $q^{\ast}$ is found such
that
\[ R_{i}(q^{\ast}) \leq T_{i} \]
The worst case response time of $\tau_{i}$ is then

\begin{equation}
R_{i} = \max_{q = 0,1,\ldots,q^{\ast}} R_{i}(q)
\end{equation}

For more details the reader is directed to \cite{Burns&96}.

\section{Deadline monotonic priority assignment\label{sec:priorities}}

It has been proved that for a large subset of the computation model,
deadline-monotonic priorirty assignment is optimal
\cite{Leung&82}. The RTA tool assigns priorities to tasks in deadline
monotonic order unless prevented by the \verb'-p' option (see
\ref{sec:options}).

\section{Features not yet supported}

\begin{changebar}
The effect of the run-time environment (e.g. context switch, interrupt
handling, clock handler, etc.) is not included in the current version
of the RTA tool.

Other features which will be added on subsequent releases of the RTA
program include support for alternative locking protocols and priority
assignment algorithms, as well as tasks with different criticality values.
\end{changebar}

% -------- Appendices ------------------------------------
\appendix

\chapter{Installing RTA\label{ch:installation}}

\section{Retrieving RTA}

The RTA installation files can be found on the DIT/UPM ftp server, at the URL:
\begin{alltt}
   ftp://ftp.dit.upm.es/str/software/rta/
\end{alltt}

The \verb'pub/str/software/rta' directory at \verb'ftp.dit.upm.es'
contains several files:

\begin{itemize}
\item A README file with instructions for downloading and unpacking
RTA distributions.
\item A file called COPYING with a copy of the GNU Public License.
\item Binary distribution files for unix platforms, containing
\verb'tar' archives compresed with \verb'gzip'.%
\footnote{% \texttt{gzip} is a free compressino utility which can be
found at GNU servers.}
\item A binary distribution file for win32 platforms, containing a
\verb'tar' archive compressed with \verb'gzip'.
\item A source distribution file, containing a \verb'tar' archive
compressed with \verb'gzip'.
\item A documentation file containing copies of this guide in several
formats.
\end{itemize}

Since RTA is written in Ada, you need an Ada~95 compiler in order to build RTA
from the sources. There is a free compiler, GNAT, which can be dowloaded from
\verb'http://www.gnat.com' and many mirror sites.

Make sure that you download the files with ftp in binary mode.

\section{Installing or building an RTA distribution}

Intructions for installing and running RTA are in the README files
which come with each distribution.  The source distribution also
contains instructions for building RTA.

%\chapter{GNU General Public License\label{ch:license}}

%\begin{center}
%{\Large\textbf{GNU GENERAL PUBLIC LICENSE}}\\
%		       Version 2, June 1991\\

% Copyright (C) 1989, 1991 Free Software Foundation, Inc.
% 59 Temple Place - Suite 330, Boston, MA 02111-1307, USA\\[2\baselineskip]
% Everyone is permitted to copy and distribute verbatim copies
% of this license document, but changing it is not allowed.
%\end{center}

%\section*{Preamble}

%  The licenses for most software are designed to take away your
%freedom to share and change it.  By contrast, the GNU General Public
%License is intended to guarantee your freedom to share and change free
%software--to make sure the software is free for all its users.  This
%General Public License applies to most of the Free Software
%Foundation's software and to any other program whose authors commit to
%using it.  (Some other Free Software Foundation software is covered by
%the GNU Library General Public License instead.)  You can apply it to
%your programs, too.

%  When we speak of free software, we are referring to freedom, not
%price.  Our General Public Licenses are designed to make sure that you
%have the freedom to distribute copies of free software (and charge for
%this service if you wish), that you receive source code or can get it
%if you want it, that you can change the software or use pieces of it
%in new free programs; and that you know you can do these things.

%  To protect your rights, we need to make restrictions that forbid
%anyone to deny you these rights or to ask you to surrender the rights.
%These restrictions translate to certain responsibilities for you if you
%distribute copies of the software, or if you modify it.

%  For example, if you distribute copies of such a program, whether
%gratis or for a fee, you must give the recipients all the rights that
%you have.  You must make sure that they, too, receive or can get the
%source code.  And you must show them these terms so they know their
%rights.

%  We protect your rights with two steps: (1) copyright the software, and
%(2) offer you this license which gives you legal permission to copy,
%distribute and/or modify the software.

%  Also, for each author's protection and ours, we want to make certain
%that everyone understands that there is no warranty for this free
%software.  If the software is modified by someone else and passed on, we
%want its recipients to know that what they have is not the original, so
%that any problems introduced by others will not reflect on the original
%authors' reputations.

%  Finally, any free program is threatened constantly by software
%patents.  We wish to avoid the danger that redistributors of a free
%program will individually obtain patent licenses, in effect making the
%program proprietary.  To prevent this, we have made it clear that any
%patent must be licensed for everyone's free use or not licensed at all.

%  The precise terms and conditions for copying, distribution and
%modification follow.

%\section*{
%   TERMS AND CONDITIONS FOR COPYING, DISTRIBUTION AND MODIFICATION
%}

%\begin{enumerate}\setcounter{enumi}{-1}
%\item This License applies to any program or other work which contains
%a notice placed by the copyright holder saying it may be distributed
%under the terms of this General Public License.  The "Program", below,
%refers to any such program or work, and a "work based on the Program"
%means either the Program or any derivative work under copyright law:
%that is to say, a work containing the Program or a portion of it,
%either verbatim or with modifications and/or translated into another
%language.  (Hereinafter, translation is included without limitation in
%the term "modification".)  Each licensee is addressed as "you".

%Activities other than copying, distribution and modification are not
%covered by this License; they are outside its scope.  The act of
%running the Program is not restricted, and the output from the Program
%is covered only if its contents constitute a work based on the
%Program (independent of having been made by running the Program).
%Whether that is true depends on what the Program does.

%\item You may copy and distribute verbatim copies of the Program's
%source code as you receive it, in any medium, provided that you
%conspicuously and appropriately publish on each copy an appropriate
%copyright notice and disclaimer of warranty; keep intact all the
%notices that refer to this License and to the absence of any warranty;
%and give any other recipients of the Program a copy of this License
%along with the Program.

%You may charge a fee for the physical act of transferring a copy, and
%you may at your option offer warranty protection in exchange for a fee.

%  \item You may modify your copy or copies of the Program or any portion
%of it, thus forming a work based on the Program, and copy and
%distribute such modifications or work under the terms of Section 1
%above, provided that you also meet all of these conditions:

%    \begin{enumerate}
%    \item You must cause the modified files to carry prominent notices
%    stating that you changed the files and the date of any change.

%    \item You must cause any work that you distribute or publish, that in
%    whole or in part contains or is derived from the Program or any
%    part thereof, to be licensed as a whole at no charge to all third
%    parties under the terms of this License.

%    \item If the modified program normally reads commands interactively
%    when run, you must cause it, when started running for such
%    interactive use in the most ordinary way, to print or display an
%    announcement including an appropriate copyright notice and a
%    notice that there is no warranty (or else, saying that you provide
%    a warranty) and that users may redistribute the program under
%    these conditions, and telling the user how to view a copy of this
%    License.  (Exception: if the Program itself is interactive but
%    does not normally print such an announcement, your work based on
%    the Program is not required to print an announcement.)
%    \end{enumerate}

%These requirements apply to the modified work as a whole.  If
%identifiable sections of that work are not derived from the Program,
%and can be reasonably considered independent and separate works in
%themselves, then this License, and its terms, do not apply to those
%sections when you distribute them as separate works.  But when you
%distribute the same sections as part of a whole which is a work based
%on the Program, the distribution of the whole must be on the terms of
%this License, whose permissions for other licensees extend to the
%entire whole, and thus to each and every part regardless of who wrote it.

%Thus, it is not the intent of this section to claim rights or contest
%your rights to work written entirely by you; rather, the intent is to
%exercise the right to control the distribution of derivative or
%collective works based on the Program.

%In addition, mere aggregation of another work not based on the Program
%with the Program (or with a work based on the Program) on a volume of
%a storage or distribution medium does not bring the other work under
%the scope of this License.

%\item You may copy and distribute the Program (or a work based on it,
%under Section 2) in object code or executable form under the terms of
%Sections 1 and 2 above provided that you also do one of the following:

%    \begin{enumerate}
%    \item Accompany it with the complete corresponding machine-readable
%    source code, which must be distributed under the terms of Sections
%    1 and 2 above on a medium customarily used for software interchange; or,

%    \item Accompany it with a written offer, valid for at least three
%    years, to give any third party, for a charge no more than your
%    cost of physically performing source distribution, a complete
%    machine-readable copy of the corresponding source code, to be
%    distributed under the terms of Sections 1 and 2 above on a medium
%    customarily used for software interchange; or,

%    \item Accompany it with the information you received as to the offer
%    to distribute corresponding source code.  (This alternative is
%    allowed only for noncommercial distribution and only if you
%    received the program in object code or executable form with such
%    an offer, in accord with Subsection b above.)
%    \end{enumerate}

%The source code for a work means the preferred form of the work for
%making modifications to it.  For an executable work, complete source
%code means all the source code for all modules it contains, plus any
%associated interface definition files, plus the scripts used to
%control compilation and installation of the executable.  However, as a
%special exception, the source code distributed need not include
%anything that is normally distributed (in either source or binary
%form) with the major components (compiler, kernel, and so on) of the
%operating system on which the executable runs, unless that component
%itself accompanies the executable.

%If distribution of executable or object code is made by offering
%access to copy from a designated place, then offering equivalent
%access to copy the source code from the same place counts as
%distribution of the source code, even though third parties are not
%compelled to copy the source along with the object code.

%\item You may not copy, modify, sublicense, or distribute the Program
%except as expressly provided under this License.  Any attempt
%otherwise to copy, modify, sublicense or distribute the Program is
%void, and will automatically terminate your rights under this License.
%However, parties who have received copies, or rights, from you under
%this License will not have their licenses terminated so long as such
%parties remain in full compliance.

%\item You are not required to accept this License, since you have not
%signed it.  However, nothing else grants you permission to modify or
%distribute the Program or its derivative works.  These actions are
%prohibited by law if you do not accept this License.  Therefore, by
%modifying or distributing the Program (or any work based on the
%Program), you indicate your acceptance of this License to do so, and
%all its terms and conditions for copying, distributing or modifying
%the Program or works based on it.

%\item  Each time you redistribute the Program (or any work based on the
%Program), the recipient automatically receives a license from the
%original licensor to copy, distribute or modify the Program subject to
%these terms and conditions.  You may not impose any further
%restrictions on the recipients' exercise of the rights granted herein.
%You are not responsible for enforcing compliance by third parties to
%this License.

%\item  If, as a consequence of a court judgment or allegation of patent
%infringement or for any other reason (not limited to patent issues),
%conditions are imposed on you (whether by court order, agreement or
%otherwise) that contradict the conditions of this License, they do not
%excuse you from the conditions of this License.  If you cannot
%distribute so as to satisfy simultaneously your obligations under this
%License and any other pertinent obligations, then as a consequence you
%may not distribute the Program at all.  For example, if a patent
%license would not permit royalty-free redistribution of the Program by
%all those who receive copies directly or indirectly through you, then
%the only way you could satisfy both it and this License would be to
%refrain entirely from distribution of the Program.

%If any portion of this section is held invalid or unenforceable under
%any particular circumstance, the balance of the section is intended to
%apply and the section as a whole is intended to apply in other
%circumstances.

%It is not the purpose of this section to induce you to infringe any
%patents or other property right claims or to contest validity of any
%such claims; this section has the sole purpose of protecting the
%integrity of the free software distribution system, which is
%implemented by public license practices.  Many people have made
%generous contributions to the wide range of software distributed
%through that system in reliance on consistent application of that
%system; it is up to the author/donor to decide if he or she is willing
%to distribute software through any other system and a licensee cannot
%impose that choice.

%This section is intended to make thoroughly clear what is believed to
%be a consequence of the rest of this License.

%\item  If the distribution and/or use of the Program is restricted in
%certain countries either by patents or by copyrighted interfaces, the
%original copyright holder who places the Program under this License
%may add an explicit geographical distribution limitation excluding
%those countries, so that distribution is permitted only in or among
%countries not thus excluded.  In such case, this License incorporates
%the limitation as if written in the body of this License.

%\item The Free Software Foundation may publish revised and/or new versions
%of the General Public License from time to time.  Such new versions will
%be similar in spirit to the present version, but may differ in detail to
%address new problems or concerns.

%Each version is given a distinguishing version number.  If the Program
%specifies a version number of this License which applies to it and "any
%later version", you have the option of following the terms and conditions
%either of that version or of any later version published by the Free
%Software Foundation.  If the Program does not specify a version number of
%this License, you may choose any version ever published by the Free Software
%Foundation.

%\item If you wish to incorporate parts of the Program into other free
%programs whose distribution conditions are different, write to the author
%to ask for permission.  For software which is copyrighted by the Free
%Software Foundation, write to the Free Software Foundation; we sometimes
%make exceptions for this.  Our decision will be guided by the two goals
%of preserving the free status of all derivatives of our free software and
%of promoting the sharing and reuse of software generally.

%\newpage

%{\Large \textbf{NO WARRANTY}}

%\item BECAUSE THE PROGRAM IS LICENSED FREE OF CHARGE, THERE IS NO WARRANTY
%FOR THE PROGRAM, TO THE EXTENT PERMITTED BY APPLICABLE LAW.  EXCEPT WHEN
%OTHERWISE STATED IN WRITING THE COPYRIGHT HOLDERS AND/OR OTHER PARTIES
%PROVIDE THE PROGRAM "AS IS" WITHOUT WARRANTY OF ANY KIND, EITHER EXPRESSED
%OR IMPLIED, INCLUDING, BUT NOT LIMITED TO, THE IMPLIED WARRANTIES OF
%MERCHANTABILITY AND FITNESS FOR A PARTICULAR PURPOSE.  THE ENTIRE RISK AS
%TO THE QUALITY AND PERFORMANCE OF THE PROGRAM IS WITH YOU.  SHOULD THE
%PROGRAM PROVE DEFECTIVE, YOU ASSUME THE COST OF ALL NECESSARY SERVICING,
%REPAIR OR CORRECTION.

%\item IN NO EVENT UNLESS REQUIRED BY APPLICABLE LAW OR AGREED TO IN WRITING
%WILL ANY COPYRIGHT HOLDER, OR ANY OTHER PARTY WHO MAY MODIFY AND/OR
%REDISTRIBUTE THE PROGRAM AS PERMITTED ABOVE, BE LIABLE TO YOU FOR DAMAGES,
%INCLUDING ANY GENERAL, SPECIAL, INCIDENTAL OR CONSEQUENTIAL DAMAGES ARISING
%OUT OF THE USE OR INABILITY TO USE THE PROGRAM (INCLUDING BUT NOT LIMITED
%TO LOSS OF DATA OR DATA BEING RENDERED INACCURATE OR LOSSES SUSTAINED BY
%YOU OR THIRD PARTIES OR A FAILURE OF THE PROGRAM TO OPERATE WITH ANY OTHER
%PROGRAMS), EVEN IF SUCH HOLDER OR OTHER PARTY HAS BEEN ADVISED OF THE
%POSSIBILITY OF SUCH DAMAGES.
%\end{enumerate}

%\vspace{2\baselineskip}

%{\Large \textbf{END OF TERMS AND CONDITIONS}}

%\newpage
%\section*{How to Apply These Terms to Your New Programs}

%  If you develop a new program, and you want it to be of the greatest
%possible use to the public, the best way to achieve this is to make it
%free software which everyone can redistribute and change under these terms.

%  To do so, attach the following notices to the program.  It is safest
%to attach them to the start of each source file to most effectively
%convey the exclusion of warranty; and each file should have at least
%the "copyright" line and a pointer to where the full notice is found.

%\begin{alltt}
%    <one line to give the program's name and a brief idea of what it does.>
%    Copyright (C) 19yy  <name of author>

%    This program is free software; you can redistribute it and/or modify
%    it under the terms of the GNU General Public License as published by
%    the Free Software Foundation; either version 2 of the License, or
%    (at your option) any later version.

%    This program is distributed in the hope that it will be useful,
%    but WITHOUT ANY WARRANTY; without even the implied warranty of
%    MERCHANTABILITY or FITNESS FOR A PARTICULAR PURPOSE.  See the
%    GNU General Public License for more details.

%    You should have received a copy of the GNU General Public License
%    along with this program; if not, write to the Free Software
%    Foundation, Inc., 59 Temple Place - Suite 330, Boston, MA 02111-1307, USA
%\end{alltt}

%Also add information on how to contact you by electronic and paper mail.

%If the program is interactive, make it output a short notice like this
%when it starts in an interactive mode:

%\begin{alltt}
%    Gnomovision version 69, Copyright (C) 19yy name of author
%    Gnomovision comes with ABSOLUTELY NO WARRANTY; for details type `show w'.
%    This is free software, and you are welcome to redistribute it
%    under certain conditions; type `show c' for details.
%\end{alltt}

%The hypothetical commands `show w' and `show c' should show the appropriate
%parts of the General Public License.  Of course, the commands you use may
%be called something other than `show w' and `show c'; they could even be
%mouse-clicks or menu items--whatever suits your program.

%You should also get your employer (if you work as a programmer) or your
%school, if any, to sign a "copyright disclaimer" for the program, if
%necessary.  Here is a sample; alter the names:

%\begin{alltt}
%  Yoyodyne, Inc., hereby disclaims all copyright interest in the program
%  `Gnomovision' (which makes passes at compilers) written by James Hacker.

%  <signature of Ty Coon>, 1 April 1989
%  Ty Coon, President of Vice
%\end{alltt}

%This General Public License does not permit incorporating your program into
%proprietary programs.  If your program is a subroutine library, you may
%consider it more useful to permit linking proprietary applications with the
%library.  If this is what you want to do, use the GNU Library General
%Public License instead of this License.

\bibliography{realtime}
\addcontentsline{toc}{chapter}{Bibliography}

\end{document}